\documentclass[a4paper]{article}

\usepackage[T1]{fontenc}
\usepackage[adobe-utopia]{mathdesign}
\usepackage[margin=3cm]{geometry}
\linespread{1.7}

\usepackage[acronym,shortcuts]{glossaries}
% This file is maintained in my notes repository. Please do not edit, all
% changes will be lost.
%
% The acronym package requires hyphenation of all expanded acronyms, the
% glossaries package does not. Therefore I use glossaries. 
%
% Use \usepackage[acronym,shortcuts]{glossaries} to get commands similar to the
% acronyms package. The main difference is that the acronyms should be included
% after \begin{document}

\newacronym{AAR}{AAR}{adaptive autoregressive}
\newacronym{ADHD}{ADHD}{attention deficit hyperactivity disorder}
\newacronym{AIM}{AIM}{automatic intention mapping}
\newacronym{ALS}{ALS}{amyotrophic lateral sclerosis}
\newacronym{API}{API}{application programming interface}
\newacronym{AT}{AT}{assistive technology}
\newacronym{AUC}{AUC}{area under the curve}
\newacronym{BCI}{BCI}{brain-computer interface}
\newacronym{BP}{BP}{Bereitschaftspotential}
\newacronym{CAI}{CAI}{Cognitive Artificial Intelligence}
\newacronym{CAR}{CAR}{common average reference}
\newacronym{CCA}{CCA}{canonical correlation analysis}
\newacronym{CFC}{CFC}{cross-frequency coupling}
\newacronym{CRF}{CRF}{conditional random field}
\newacronym{CSP}{CSP}{common spatial patterns}
\newacronym{DBN}{DBN}{deep belief network}
\newacronym{DCC}{DCC}{Donders Center for Cognition}
\newacronym{DCCN}{DCCN}{Donders Center for Cognitive Neuroimaging}
\newacronym{DCM}{DCM}{dynamic causal modeling}
\newacronym{DEDICOM}{DEDICOM}{decomposition into directional components}
\newacronym{DFT}{DFT}{discrete Fourier transform}
\newacronym{DI}{DI}{Donders Institute}
\newacronym{dSVM}{dSVM}{dependent-samples support vector machine}
\newacronym{DTI}{DTI}{diffusion tensor imaging}
\newacronym{ECoG}{ECoG}{electrocorticography}
\newacronym{EEG}{EEG}{electroencephalography}
\newacronym{EMG}{EMG}{electromyography}
\newacronym{EOG}{EOG}{electroocculography}
\newacronym{ERD}{ERD}{event-related desynchronization}
\newacronym{ERP}{ERP}{event-related potential}
\newacronym{ERS}{ERS}{event-related synchronization}
\newacronym{EWMA}{EMWA}{exponentially weighted moving average}
\newacronym{FFT}{FFT}{fast Fourier transform}
\newacronym{FIR}{FIR}{finite impulse response}
\newacronym{fMRI}{fMRI}{functional magnetic resonance imaging}
\newacronym{fNIRS}{fNIRS}{functional near-infrared spectroscopy}
\newacronym{FSP}{FSP}{fixed spatial patterns}
\newacronym{GLM}{GLM}{general linear model}
\newacronym{GLS}{GLS}{generalized least squares}
\newacronym{GPC}{GPC}{Gaussian processes classification}
\newacronym{GPR}{GPR}{Gaussian processes regression}
\newacronym{HCI}{HCI}{human-computer interaction}
\newacronym{HMI}{HMI}{Human Media Interaction}
\newacronym{HMM}{HMM}{hidden Markov model}
\newacronym{ICA}{ICA}{independent component analysis}
\newacronym{iCSP}{iCSP}{invariant common spatial patterns}
\newacronym{IID}{iid}{independent and identically distributed}
\newacronym{IIR}{IIR}{infinite impulse response}
\newacronym{IP}{IP}{intellectual property}
\newacronym{ISI}{ISI}{inter-stimulus interval}
\newacronym{ITI}{ITI}{inter-trial interval}
\newacronym{ITR}{ITR}{information transfer rate}
\newacronym{KKT}{KKT}{Karush--Kuhn--Tucker}
\newacronym{KLD}{KLD}{Kullback-Leibler divergence}
\newacronym{KL}{KL}{Kullback-Leibler}
\newacronym{KLR}{KLR}{kernel logistic regression}
\newacronym{KLT}{KLT}{Karhunen-Lo\`eve transform}
\newacronym{KTA}{KTA}{kernel target alignment}
\newacronym{LDA}{LDA}{linear discriminant analysis}
\newacronym{LFP}{LFP}{local field potential}
\newacronym{LOC}{LOC}{loss of control}
\newacronym{logBP}{logBP}{log band-power}
\newacronym{LRP}{LRP}{lateralized readiness potential}
\newacronym{LTD}{LTD}{long-term synaptic depression}
\newacronym{LTP}{LTP}{long-term synaptic potentiation}
\newacronym{MEG}{MEG}{magnetoencephalogram}
\newacronym{MI}{MI}{mutual information}
\newacronym{ML}{ML}{machine learning}
\newacronym{MRI}{MRI}{magnetic resonance imaging}
\newacronym{MS}{MS}{multiple sclerosis}
\newacronym{NNMF}{NMF}{non-negative matrix factorization}
\newacronym{NNO}{NNO}{non-negative oscillations}
\newacronym{NNTF}{NTF}{non-negative tensor factorization}
\newacronym{PAC}{PAC}{phase-amplitude coupling}
\newacronym{PARAFAC2}{PARAFAC2}{parallel factor analysis 2}
\newacronym{PARAFAC}{PARAFAC}{parallel factor analysis}
\newacronym{PCA}{PCA}{principal component analysis}
\newacronym{PC}{PC}{principal component}
\newacronym{PLV}{PLV}{phase-locking value}
\newacronym{PSD}{PSD}{power spectral density}
\newacronym{QDA}{QDA}{quadratic discriminant analysis}
\newacronym{QLYSVM}{QLYSVM}{quadratic-loss Y-SVM}
\newacronym{QP}{QP}{quadratic programming}
\newacronym{RBM}{RBM}{restricted Boltzmann machine}
\newacronym{ROC}{ROC}{receiver operating characteristic}
\newacronym{ROI}{ROI}{region of interest}
\newacronym{SAM}{SAM}{self-assessment manikin}
\newacronym{SCP}{SCP}{slow cortical potentials}
\newacronym{SD}{SD}{subject-dependent}
\newacronym{SI}{SI}{subject-independent}
\newacronym{SMR}{SMR}{sensory motor rhythm}
\newacronym{SNR}{SNR}{signal-to-noise ratio}
\newacronym{SOB}{SOB}{second-order baseline}
\newacronym{SSA}{SSA}{stationary subspace analysis}
\newacronym{SS}{SS}{slow sphering}
\newacronym{SSVEP}{SSVEP}{steady-state visually evoked potential}
\newacronym{SVD}{SVD}{singular value decomposition}
\newacronym{SVM}{SVM}{support vector machine}
\newacronym{TDP}{TDP}{time-domain parameters}
\newacronym{UT}{UT}{University of Twente}
\newacronym{UvA}{UvA}{University of Amsterdam}
\newacronym{ZT}{ZT}{zero training}
\newacronym{RU}{RU}{Radboud University Nijmegen}


\title{Abstract PhD thesis\\ \textsc{Robust Brain-Computer Interfaces}}
\author{Boris Reuderink}

\begin{document}
\maketitle

A \ac{BCI} enables direct communication from the brain to devices,
bypassing the traditional pathway of peripheral nerves and muscles.   
%
Current \acp{BCI} aimed at patients require that the user invests weeks, or even
months, to learn the skill to intentionally modify their brain signals. This can
be reduced to a calibration session of about half an hour per session if
\ac{ML} methods are used. The laborious recalibration is still needed due to
inter-session differences in the statistical properties of the \ac{EEG} signal.
%
Further, the natural variability in spontaneous \ac{EEG} violates basic
assumptions made by the \ac{ML} methods used to train the \ac{BCI} classifier,
and causes the classification accuracy to fluctuate unpredictably. These
fluctuations make the current generation of \acp{BCI} unreliable.
%
In this dissertation, we will investigate the nature of these variations in the
\ac{EEG} distributions, and introduce two new, complementary methods to
overcome these two key issues.
%
To confirm the problem of non-stationary brain signals, we first show that
\acp{BCI} based on commonly used signal features are sensitive to changes in
the mental state of the user.
% 
We proceed by describing a method aimed at removing these changes in signal
feature distributions. We have devised a method that uses a \ac{SOB} to
specifically isolate these relative changes in neuronal firing synchrony. To
the best of our knowledge this is the first \ac{BCI} classifier that works on
out-of-sample subjects without any loss of performance.
%
Still, the assumption made by \ac{ML} methods that the training data consists of
samples that are \ac{IID} is violated, because \ac{EEG} samples nearby in time
are highly correlated. Therefore we derived a generalization of the well-known
\ac{SVM} classifier, that takes the resulting chronological structure of
classification errors into account. Both on artificial data and real \ac{BCI}
data, overfitting is reduced with this \ac{dSVM}, leading to \acp{BCI} with an
increased information throughput.

%With the \ac{SOB} features we have addressed the first key issue (of investment
%of time) for a motor imagery task. The second key issue is partially addressed with the \ac{dSVM}, but raises new questions regarding the
%interpretation of \ac{BCI} performance. 
\end{document}
