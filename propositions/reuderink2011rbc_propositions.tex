\documentclass[a5paper]{article}
\usepackage[margin=1.8cm]{geometry} % needed to actually *change* papersize

\usepackage[utf8]{inputenc}
\usepackage[T1]{fontenc}
\usepackage[adobe-utopia]{mathdesign}

\title{Propositions}
  %\\\textsc{Robust Brain-Computer Interfaces}}
\author{Boris Reuderink}

\begin{document}
\maketitle
\thispagestyle{empty}

\begin{enumerate}
\item The assumption in machine learning and statistics that our observations
are \emph{independently and identically distributed} is a lie that we need in order to do our research (Chapter 4).

\item While the BCI community is generally opposed to hyping the technology in
the media, the field often contributes to unrealistic expectations by using
improper evaluation methodologies.

\item Multiple-test correction is used to control false positives when a series
of statistical tests is performed for a single hypothesis. This is only
sensible if one corrects also for all the fruitless tests performed on the same
dataset.

\item Much of the observed ``non-stationary'' BCI classification performance
within sessions is caused by the researchers interpreting structure in random
noise.

\item It would be beneficial if the obligatory peer review for articles was
extended to review software and scripts developed for the article as well.

\item The Kolmogorov complexity of a PhD thesis in a technical field is
inversely related to its quality.

\item Many insights from machine learning apply to our daily lives as well; for
example, most researchers have issues with the non-differentiable cost associated with missing a deadline.

\item Hoewel een computer een uitstekend medium is om ideeën op uit te wer\-ken, ontstaan deze ideeën meestal niet achter een beeldscherm.

\item Ouderschap vormt gek genoeg geen belemmering voor het werken aan een
proefschrift; blijkbaar is slaaptekort niet cumulatief.

%\item iets met simpel

%\item emotions?
\end{enumerate}
\end{document}
