\cleartoevenpage
\vspace*{1.5cm}
\begin{verse} \it
Into the distance, a ribbon of black\\
Stretched to the point of no turning back\\
A flight of fancy on a windswept field\\
Standing alone my senses reeled\\
A fatal attraction is holding me fast,\\
How can I escape this irresistible grasp?

Can't keep my eyes from the circling skies\\
Tongue-tied and twisted, just an earth-bound misfit, I

Ice is forming on the tips of my wings\\
Unheeded warnings, I thought, I thought of everything\\
No navigator to find my way home\\
Unladened, empty and turned to stone\\
A soul in tension --- that's learning to fly\\
Condition grounded but determined to try

Can't keep my eyes from the circling skies\\
Tongue-tied and twisted just an earth-bound misfit, I

Above the planet on a wing and a prayer,\\
My grubby halo, a vapour trail in the empty air,\\
Across the clouds I see my shadow fly\\
Out of the corner of my watering eye\\
A dream unthreatened by the morning light\\
Could blow this soul right through the roof of the night

There's no sensation to compare with this\\
Suspended animation, a state of bliss

Can't keep my mind from the circling skies\\
Tongue-tied and twisted just an earth-bound misfit, I \\
\vspace*{1em}
\hspace*{10em} --- Pink Floyd, ``Learning to Fly''
\end{verse}




\chapter{Preface}
% my experience
\lettrine{I}{} have had a long-standing interest in how the human brain works.
When at the end of my master's education the option arose to conduct research on
\acp{BCI}, I had no choice but to pursue. The very intense past four years
that followed culminated in this little book. This period was marked by great
freedom to explore and follow intellectual curiosity, but also required
perseverance, reflection on my strengths and weaknesses --- and hard work.
Looking back, I think of this period as a very valuable, and defining period of
my life.

% what we did in four years
When I started four years ago, the \ac{BCI} research was just starting at our
\ac{HMI} group, funded by the national BrainGain project. Through collaboration
with Peter Desain's \ac{CAI} group in Nijmegen\todo{Lynn: oude naam CAI
noemen}, we quickly found our way in the field of \acp{BCI}. Over time, we
developed our own best practises, and performed \ac{BCI} research from the
\ac{HCI} perspective. Simultaneously, I developed open source packages for
signal processing and machine learning that powered many of our \ac{BCI} demos.

% HMI
Lots of people contributed to this great experience. First of all, I would like
to thank the people from the \ac{HMI} group. Specifically, my promotor Anton
Nijholt for providing a place within the \ac{HMI} group and allowing me to
deviate from the sharp boundaries of the project assignment, and my daily
supervisor Mannes Poel, who was always available for work-related reflections
and kept an eye not only on the progress of my work, but was also interested in
my personal well-being. During my travel to the university I often spoke with
Dirk Heylen, who often stimulated me to reflect more deeply on my statements. I
would also like to thank the secretaries Charlotte and Alice for their support,
and thank Lynn for proofreading all my papers.
%
Ronald shared my view on machine learning, and provided a listening ear for my
doubts regarding the field. Dennis provided the occasional odd thought, folk
music and general serendipity, and pushed me to think of the wider implications
of my work.

% BrainMedia
I would like to thank the BrainMedia subgroup for the pleasant discussions and
reflections on \acp{BCI}, and the experiences we shared. In particular, I would
like to thank my roommate Christian with whom I shared the doubts and worries
of obtaining a PhD, but also the successes, and Danny for our shared
development of a vision on practical \acp{BCI}, her optimistic view, and our collaborations.

% CAI
This work would not be the same without my weekly visits to the \ac{CAI} group.
Talking to Jason taught me that a simple mathematical proof can often save
literally months of empirical research, and I learned a great variety of
elegant (\acl{ML}) tricks. When visiting, Rutger always had time for a creative
brainstorm, and was as motivated as I am to improve the practise, and not only
the theory of \acp{BCI}. I greatly enjoyed my collaboration with this group,
and I would like to thank Peter Desain for creating many opportunities for me.

% Friends and family
I know that the last years have been hard for the people around me. I would
like to thank my friends and family for their support and acceptance of my
sometimes lacking focus. My special thanks and admiration go to my wife Sanne
and my daughter Lauren for enabling me to perform this research.
