\section{Conclusions and future work}
\begin{sloppypar}
We have presented an \ac{SOB} procedure that reduces inter-subject and
inter-session variability, and demonstrated that \ac{SOB}-covariance features
allow for cross-subject motor imagery classification without a loss of
performance compared to within-subject classification with the popular \ac{CSP}
based \ac{BCI} classifier.
%
The advantage of the \ac{SOB} based covariance features is that they are robust
against inter-session and inter-subject variation, and that standard
classifiers such as the \ac{SVM} can be used without the need of any adaptation
or post-processing of the outputs (e.g. bias-adaptation).
%
Furthermore, the online processing is simplified as it can be implemented as a
stateless, fixed pipeline that does not handle the incoming data differently
during a calibration session.

In addition to the practical advantages of removing the need for the laborious
calibration sessions, changing from \acl{SD} to \acl{SI} \acp{BCI} also
simplifies multi-disciplinary \ac{BCI} research. It allows researchers to work
with validated \ac{BCI} classifiers that are known to work with a certain
probability on the target population, and focusses on the intended brain
regions.
% 
The development of \acl{SI} \acp{BCI} can facilitate new applications for which
collecting enough subject-specific training data before each session is not
feasible, such as for example fatigue detection, screening of neurological
disorders or classification of emotional states.
\end{sloppypar}

The method described uses a single, broad frequency band. For future work, the
features can be extended to include multiple frequency bands as in
\cite{lotte2009cdt, farquhar2009lfs}. Work with naive \ac{BCI} has shown that subject-specific frequency bands can dramatically improve performance \cite{blankertz2008bbc}.
 
\begin{sloppypar}
Another interesting research direction is to generalize to recordings with
different electrode layouts. As the learned covariance weights were quite
sparse, the correspondence of a few key sensor locations might be enough to
generalize to new sensor configurations. This sparsity in covariance space
seems to suggest that it is more natural to think in covariance (or coherence)
between brain regions, than to think of power in spatially filtered sources.
The second row in Figure~\ref{fig:ud_patt} shows spatial filters with a dipolar
structure on the motor cortices. The combination of these and many more filters
is needed to isolate a specific covariance pair --- that is, the intriguing
dipolar structure is probably residue caused by the decomposition of the weight
matrix.
\end{sloppypar}

Finally, although the method presented works causally, it should be validated in
an online experiment with a user in the loop.
