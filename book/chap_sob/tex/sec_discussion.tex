\section{Discussion}
The results indicate that the new \acl{SOB} covariance features provide a
robust alternative to \ac{CSP} features for classification of motor imagery,
and generalize to new, unseen subjects without additional calibration or
training. Apparently, the normalization performed with the \ac{SOB} removes
enough inter-subject variability to generalize to new subjects.
%
However, the dataset used in this research contains rather few trials, hence
the \ac{SD} \ac{CSP} performance might have suffered from insufficient training
data. Nevertheless, recording more trials is not always an option, and the
\ac{SD} performance obtained in our study is similar to scores presented in
\cite{fazli2009sim} where much longer sessions were used. Further, when a
similar \ac{SD} \ac{CSP} pipeline with a broadband spectral filter was applied
to naive (i.e. users that have not used a \ac{BCI} before) users in
\cite{blankertz2008bbc}, the results were barely above chance although
280 trials were available for evaluation. The performance increased
dramatically though when subject-specific \emph{spectral} filters were used. 

The \ac{SOB}'s $\alpha$ parameter seems of some importance for generalization
over subjects. While for \ac{SD} classification a long half-life was preferred,
$\alpha$'s with a short half-life were preferred for \ac{SI} classification.
%
Presumably, slow adaptation is preferred for \ac{SD} classification because it
allows the classifier to model and exploit session-specific variations, such as
for example bad channels and \ac{EEG} artifacts. For \ac{SI} classifications
modeling these variations is generally not helpful as they are not consistent
over subjects. Shorter half-lives reduce these variations, and are thus
preferred for \ac{SI} classification.
 
\begin{sloppypar}
It is noteworthy to mention that the best $\alpha$-value for
\ac{SOB}-covariance features was selected based on the performance on
the test subjects. This might slightly overestimate the true performance. 
%
Usually these hyper-parameters (e.g. the \ac{SVM}'s $c$-parameter) are set
based on performance estimates obtained with cross-validation. The half-life
constant $\alpha$ could be chosen with cross-validation, but since the \ac{SOB}
is a preprocessing method this is often computationally impractical.
%
Since current \ac{BCI} pipelines have several preprocessing hyper-parameters
that are fixed a priori (e.g. the cut-off values in the band-pass filter, or
the $m=6$ spatial filters), we can imagine that a fixed $\alpha$ can be used as
well. Given that even the worst $\alpha$ performs better than the alternatives
in \ac{SI} classification, the performance gain seems fairly robust for a wide
range of $\alpha$-values (Fig.~\ref{fig:csob_acc}). Therefore, we expect that
choosing $\alpha=0.16$ (a half-life of 4 trials) a priori will be adequate in
practice. This value is probably independent of the mental task used.
\end{sloppypar}
