\section{Discussion}
We successfully induced a negative mental state with lack of feeling of control
in the \ac{LOC} condition, as shown by the significant differences in the
\ac{SAM} ratings.
% 
The performance of our \ac{ERD} classifier improved significantly in the
\ac{LOC} condition. However, we did find small differences in the statistics of
known confounding behaviour. After correcting for these factors, there still
were differences in the performance of the \ac{CSP}-based \ac{BCI}. This
difference in performance is probably caused by a change in feature
distribution associated with the change in the mental state of the subject.

% lower emotional contrast due to freq. match.
The correction for confounding statistics could have reduced the contrast
between the experimental conditions, since changes in behaviour and emotional
state are interrelated and the behavioural changes were removed. This might
explain why the over the over-subjects effect disappeared. After correction,
more within-subject differences became significant, sometimes in opposing directions.
%
While correcting for confounding factors was necessary to exclude the
possibility that the effects depend solely on the \ac{LOC} induction method,
the results without correction are more ecologically valid, since increased
effort and repeated attempts to perform the same action are to be expected when
the \ac{BCI} fails.

% improvement due to nonstat?
It is surprising that the same \ac{ERD} classifier predicts the labels of data
from a different distribution significantly better it predicts labels of the
data with the distribution it was trained on. The significant differences in
the \ac{AUC} values indicates that the improvement in the \ac{LOC} condition
cannot be simply explained by a bias shift, since that would not account for improved ranking of trials. This suggests that the \ac{ERD}
feature distributions of the two classes move away from the classifier's
hyperplane in the \ac{LOC} condition.

% comparison with Jatzev
Our findings for the \ac{ERD} classifiers contradict the findings presented in
\cite{jatzev2008ecn}, where a \emph{decrease} in performance for the \ac{LOC}
condition was reported.   
%
On \ac{ERP} classification the studies do agree; \cite{jatzev2008ecn}
reports no significant difference, and also in our study no significant effect
was found for \ac{LRP} classification
%
The differences between our study and \cite{jatzev2008ecn} for \ac{ERD}
classification could be caused by several differences in the experimental
setup: a) our \ac{ITI} is about half the \ac{ITI} used in \cite{jatzev2008ecn},
b) we did gather training and evaluation data in the same environment with the
same user tasks, c) our subjects were not informed that the system would
respond incorrectly, and d) unlike \cite{jatzev2008ecn}, we show that the
performance differences between the two conditions were not (fully) caused by
behavioural differences. But both our study and \cite{jatzev2008ecn} do agree
that \ac{LOC} has a profound influence on the \ac{ERD} features.

