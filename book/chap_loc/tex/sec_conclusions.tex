\section{Conclusions and future work}
% summary
\begin{sloppypar}
In this chapter, we have presented an experiment in which a simulated 
non-responding \ac{BCI} controller was used to study whether changes
in the user's mental state have an influence on the \ac{BCI} performance. The
self-reported emotional ratings confirmed that the \ac{LOC} condition induced a
more negative, and less dominant mental state. These different mental states
were accompanied with minor behavioural changes for which we corrected the
analysis.
\end{sloppypar}

% main result
Contrary to our expectations, we observed a significant performance
\emph{increase} during the \ac{LOC} condition for the \ac{ERD} based \acp{BCI}.
For the \ac{ERP} based \acp{BCI}, we found no change in performance. The image
of a \ac{BCI} spiralling completely out of control that we sketched in the
introduction appears to be an illusion. However, the difference in performance
demonstrates that variabilities in the feature distributions related to
\ac{LOC} do in fact exist, and could be more dire under different
circumstances.

% future work
For future work in this direction, a logical next step would be to
investigate the origin of the increase of performance for \ac{ERD}
classifiers. We suspect that it might be related to a shift in attention from
the game context during normal play to the movement of the hands in the
\ac{LOC} condition. 
%
Since the strength of the beta band \ac{ERS} is related to attention in
constant isometric force motor tasks \cite{kirsteva-feige2002eap}, an increase
of attention on the motor task in the \ac{LOC} condition could result in more
pronounced beta \ac{ERD}/\ac{ERS}, and indirectly lead to better classification
results. This would form an interesting hypothesis for a follow-up study.
%
Related is also the study presented in \cite{koelewijn2008mcb} that shows a
pronounced beta rebound when the observed movement does not match the movement
the user was supposed to execute.
 
The recordings from our current experiments could also be analyzed for
correlates with emotions, as we have user-reported ratings of emotions for
every two-minute block in the experiment.  The first steps in this direction
have been taken in \cite{reuderink2012vad}. The recognition of emotions from
\ac{EEG} would be immensely valuable to both locked-in patients --- who would
otherwise have to verbalise their mood using other means, such as the P300
speller --- and to healthy users.
