\section{Previous work}
The influence of frustration associated with \ac{LOC} on a
\ac{BCI} is of great interest since it might cause the previously described
feedback loop. This influence  was previously investigated in
\cite{jatzev2008ecn, zander2009usi}. In this study, users were instructed to
use real movement with their left or right hand to rotate respectively L or
R-shaped objects to a target position in order to study the effect of \acl{LOC}
on the \ac{BCI} performance.
%
The color of the letter indicated the angle of rotation, and the user could
press a key to rotate the object in the direction indicated by the shape of the
object every second.
%
After performing a calibration block with cued left/right hand movement and two
practise blocks with this so-called RLR paradigm, a \ac{LOC} was simulated in
the third block by occasionally using a wrong angle of rotation in the
application. Both an \ac{ERD} and an \ac{ERP} based classifier were trained on
the first block, and applied to the other blocks in an off-line analysis.  A
significant difference between the training block and the \ac{LOC} block was
found for the distribution of \ac{ERD} based features, but for \ac{ERP}
features no such difference was found. This seems to indicate that there is
variability in \ac{ERD} features related to \acl{LOC}. 

However, the study described in \cite{jatzev2008ecn, zander2009usi} is lacking
on a few aspects. Most notable is the limitation that changes in \ac{BCI}
performance due to the induction of \ac{LOC}, the progression of time,
differences in stimulation and user behaviour cannot be distinguished.
%
We were interested in the influence of \ac{LOC} on the \ac{BCI} performance
independent of these other factors. Therefore we used 1) an interleaved block
design to control for effects that manifest spontaneously over time, such as
increasing fatigue, changing temperature, drying gel on the electrodes etc., 2)
we used the same environment for training and evaluating the \ac{BCI}
classifiers to minimize environmental differences not related to \ac{LOC}, 3)
we used self-reported emotional ratings to validate the effect of \acl{LOC} on
the mental state and 4) we tested and corrected for confounding behavioural
changes, such as changes in the force, speed or order of the finger movements,
and eye movements. 
