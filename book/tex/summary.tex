\chapter{Summary}
\lettrine{A}{} \ac{BCI} enables direct communication from the brain to devices,
bypassing the traditional pathway of peripheral nerves and muscles. Usually,
\acp{BCI} are targeted at paralysed patients who have no other means of
communication. The \ac{BCI} research at our \acl{HMI} group focusses on
\acp{BCI} for healthy users, especially on gaming applications, which poses
some additional requirements on the \ac{BCI}.
  
For successful \ac{BCI} adoption in general, two key issues need to be
addressed: 1) using a \ac{BCI} should be easy, and require only small
investments (of either time or money), and 2) the \ac{BCI} should be
dependable, that is to say it should function predictably with a known
accuracy.
%
In addition, a \ac{BCI} should be applied such that it provides something 
unique (e.g. a covert measure of attention), since \acp{BCI} cannot yet compete
on reliability and speed with existing input devices for non-patients. 

\begin{sloppypar}
The first key issue excludes the possibility that brain signals are recorded
with expensive medical brain-imaging equipment. Fortunately, relatively cheap
consumer hardware with semi-dry \ac{EEG} sensors is already commercially
available, solving the problem of monetary investments to a large extent. What
remains is that non-paralysed users are probably not willing to invest weeks or
even months to learn the skill to intentionally modify their brain signals,
which is common practice with traditional \acp{BCI} aimed at patients.
\Acp{BCI} based on machine learning already reduce the problem of time
investment from weeks to minutes, through automatic recognition of the user's
naturally occurring brain signals. But still, a demanding and error-prone
calibration session (in which the user is forced to demonstrate mental tasks)
is required before each use of the \ac{BCI}. This calibration session is needed
because the electrical brain signals vary from session to session. Removing this
need of a repeated training session would greatly expand the applicability of
\acp{BCI}, and lower the barrier to entry.
\end{sloppypar}

The second key issue is related to the fundamental problem that is known in the
\ac{BCI} field as non-stationary signals. The inherent variable nature of
spontaneous \ac{EEG} causes changes in the features that the \ac{BCI} uses to
detect and classify brain signals. This variability violates basic assumptions
made by the \ac{ML} methods used to train the \ac{BCI} classifier, and causes
the classification accuracy to fluctuate unpredictably. These fluctuations make
the current generation of \acp{BCI} unreliable.

Both key issues are related; both the inter-session variability and the
unreliability stem from not fully understood properties of fluctuation in the
neuronal signal's feature distributions over time, sessions, and subjects. In
this dissertation, we will investigate the nature of these variations in the
\ac{EEG} distributions, and introduce two new, complementary methods that we
have devised to overcome these two key issues.
 
To confirm the problem of non-stationary brain signals, we first show that
\acp{BCI} based on commonly used signal features are sensitive to changes in
the mental state of the user.
% 
We proceed by describing a method aimed at removing these changes in signal
feature distributions. Based on the insight that a large class of \acp{BCI} is
based on \emph{relative changes} in spectral power, but uses absolute power for
classification, we have devised a method that uses a \ac{SOB} to specifically
isolate these relative changes in neuronal firing synchrony. To the best of our
knowledge this is the first \ac{BCI} classifier that works on out-of-sample
subjects without any loss of performance.
%
With these \ac{SOB} features, we have effectively bypassed the problem of
slowly changing non-stationary distributions. Still, the assumption made by
\ac{ML} methods, that the training data contains samples that are \ac{IID}, is
violated, because \ac{EEG} samples nearby in time are highly correlated. This
chronological structure is especially troublesome during the training of the
classifier, since it may lead to overfitting due to an overestimation of the
amount of independent information present in the calibration session.
%
We derived a generalization of the well-known \ac{SVM} classifier, that takes
the chronological structure of classification errors into account. Both on
artificial data and real \ac{BCI} data, overfitting is reduced with this
\ac{dSVM}, leading to \acp{BCI} with an increased information throughput.

With the \ac{SOB} features we have addressed the first key issue (of investment
of time) for a motor imagery task. It is likely that this approach also allows
for cross-subject generalization of classifiers based on other neuronal
signatures. The second key issue is partially addressed with the \ac{dSVM}. The
method demonstrates the feasibility of modeling the relatedness of brain
signals recorded nearby in time, which is necessary to prevent overfitting in
high-dimensional feature spaces derived from \ac{EEG}. But, not assuming
\ac{IID} feature distributions raises new questions regarding the
interpretation of \ac{BCI} performance. With the two methods presented in this
dissertation, we have paved the way for a new generation of \acp{BCI} --- \acp{BCI}
that work dependably, and without the need of recalibration.
