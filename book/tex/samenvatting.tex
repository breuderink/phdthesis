\chapter{Samenvatting}
\selectlanguage{dutch} % sudo apt-get install texlive-lang-dutch
\lettrine{E}{en} brein-computer interface (BCI) maakt directe communicatie
tussen het brein en computers mogelijk, en omzeilt daarmee de traditionele
route via de zenuwen en spieren. BCI's worden meestal ontworpen voor patiënten,
die op geen enkele andere manier kunnen communiceren (door bijvoorbeeld
verlamming). Het BCI-onderzoek van onze groep richt zich echter op gezonde
gebruikers, specifiek op speltoepassingen. Deze ongebruikelijke focus leid
tot andere eisen die we aan een BCI stellen.

Voor een algemene acceptatie van BCI-technologie moeten twee kernproblemen
worden opgelost: 1) de investering die gedaan moet worden om een BCI te kunnen
gebruiken moet klein zijn voor de gebruiker (zowel financiële investeringen, als
investering van tijd), en 2) op de BCI moet vertrouwd kunnen worden; de BCI
moet dus voorspelbaar reageren, met een constante nauwkeurigheid. Daarnaast
moet een BCI natuurlijk zo worden toegepast dat het een meerwaarde biedt omdat
de huidige generatie BCI's zich nog niet kan meten met de snelheid en
betrouwbaarheid van invoerapparaten voor gezonde gebruikers.

Het eerste kernprobleem sluit de mogelijkheid uit dat hersensignalen worden
gemeten met dure medische apparatuur. Relatief goedkope consumentenhardware is
gelukkig al commercieel verkrijgbaar, wat het probleem van financiële
investeringen grotendeels oplost. Wat overblijft is het deelprobleem dat
(gezonde) gebruikers waarschijnlijk niet bereid zijn om weken of zelfs maanden
te investeren --- hetgeen gangbaar is voor BCI's gericht op patiënten --- om de
vaardigheid te leren die ze in staat stelt vrijwillig hun hersensignalen te
sturen.
  
BCI's gebaseerd op \emph{machine learning} reduceren het probleem van de te
investeren tijd van maanden naar minuten. Dit doen ze door de
persoons-afhankelijke patronen in spontane hersenactiviteit te herkennen. Maar
zelfs met deze geavanceerde BCI's is een veeleisende en foutgevoelige
kalibratiesessie, waarin de gebruiker geforceerd mentale taken moet uitvoeren,
steevast nodig voor de BCI gebruik kan worden. Deze kalibratie is nodig omdat
de hersensignalen variëren van sessie tot sessie. De toepasbaarheid en de
laagdrempeligheid van een BCI zouden sterk worden vergroot als deze
herhaaldelijke kalibratiesessie vermeden kan worden.

Het tweede kernprobleem is gerelateerd aan het fundamentele probleem dat binnen
het BCI-veld bekend staat als niet-stationaire signalen. De inherent variabele
natuur van spontane hersenactiviteit leidt tot variaties in de
signaaleigenschappen die de BCI gebruikt om hersenactiviteit te herkennen. Deze
variabiliteit schendt de basisaanname van de machine learning methodes die
gebruikt worden om een herkenner voor de BCI te trainen, en leidt tot grote
fluctuaties in de nauwkeurigheid van de herkenning van de hersenactiviteit.
Deze fluctuaties maken de huidige generatie BCI's onbetrouwbaar.

Beide kernproblemen zijn gerelateerd; zowel de inter-sessie variabiliteit als de
onbetrouwbaarheid stammen af van niet volledig begrepen variaties van de
hersensignalen over tijd, sessies en personen. In dit proefschrift onderzoeken
we de aard van deze variaties, en ontwikkelen we twee nieuwe, complementaire
technieken om deze kernproblemen op te lossen.

Om het probleem van niet-stationaire hersensignalen te bevestigen, laten we
eerst zien dat een BCI, gebaseerd op veelgebruikte signaaleigenschappen,
gevoelig is voor veranderingen in de gemoedstoestand van de gebruiker.
%
Vervolgens presenteren we een methode gericht op het verwijderen van deze
signaalveranderingen. Uitgaande van het inzicht dat een grote groep BCI's
gebaseerd is op \emph{relatieve} veranderingen in spectrale energie, maar de
absolute energie gebruikt, ontwikkelen we een methode die een tweede-orde
referentiepunt (SOB, second-order baseline) gebruikt om de relatieve
veranderingen in het vuren van neuronen te isoleren. Voor zover wij weten is
dit de eerste BCI die, zonder kalibratiesessie, spontane hersenactiviteit kan
herkennen bij nieuwe gebruikers, zonder dat dit tot prestatieverlies leidt.
%
Met deze SOB-methode hebben we het probleem van langzaam veranderende
signaaldistributies omzeild. Maar nog steeds gaat de aanname, dat voorbeelden
waarop de herkenner gebaseerd wordt onafhankelijk zijn van elkaar, niet op;
hersenactiviteit lijkt op hersenactiviteit die kort daarvoor is waargenomen. In
het bijzonder tijdens het trainen van een herkenner is het schenden van deze
aanname problematisch, omdat de hoeveelheid informatie die de voorbeelden
bevatten wordt overschat. Dit leidt tot \emph{overfitting}; het model werkt
alleen goed tijdens de kalibratiesessie.
%
Daarom hebben we een generalisatie van de bekende \ac{SVM} herkenner afgeleid,
die de chronologische structuur van herkenfouten mee kan nemen in de
optimalisatie. Zowel op kunstmatige als echte BCI-data wordt overfitting
verminderd, en leidt dit tot een verhoogde informatiedoorvoer.

Met de SOB-methode hebben we het eerste kernprobleem (investering van
tijd) opgelost voor BCI's gebaseerd op het inbeelden van bewegingen. Het is
waarschijnlijk dat deze aanpak ook voor andere mentale taken de
kalibratiesessie overbodig maakt. Het tweede kernprobleem is deels opgelost met
de generalisatie van de \ac{dSVM}: de methode demonstreert dat het mogelijk is
om de robuustheid te verbeteren door het modelleren van de structuur van de
herkenfouten. Echter, het niet aannemen van onafhankelijke observaties roept
nieuwe vragen op met betrekking tot de interpretatie van prestatiematen die
gebruikt worden om BCI's te evalueren. Met de twee methodes gepresenteerd in
deze thesis, hebben we een weg gebaand voor een nieuwe generatie BCI's --- BCI's
die betrouwbaar werken, zonder dat een kalibratiesessie nodig is.

\todo{SB: classifier ipv herkenner}
\selectlanguage{british}
