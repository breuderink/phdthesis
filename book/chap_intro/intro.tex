\chapter{Introduction} \label{chap:intro}
% TODO:
% Applications (AN)
% Distinction online and off-line
%  - in-between options
%  - impact on evaluation
%  - causality and implementation
%  - naming (real-time/online)
% Focus on machine learning, traditional off-line research on core challenges
% Non-invasive
% Impact of gaming on my research?
% Use term 'decoding'

% BCI field
\begin{sloppypar}
\lettrine{A}{} \ac{BCI} enables direct communication from the brain to devices,
bypassing the traditional pathway of peripheral nerves and muscles. In the
past, \acp{BCI} have been targeted mainly at paralysed patients or patients
with motor disabilities who have hardly any other means of communication
\cite{farwell1988tth, birbaumer1999sdp, wolpaw2002bci}. But the unique
possibilities of \ac{BCI} technology are by no means limited to those in need;
\ac{BCI} technology enables the use of signals related to attention, intentions
and mental state, without relying on indirect measures based on overt behaviour
or other physiological signals \cite{zander2011tpb, vangerven2009amp, krauledat2004isa}.
\end{sloppypar}

% types of BCI: user training vs ML
The traditional approach to \acp{BCI} is to provide the user with a device that
is controlled through a fixed function of the brain signals, and let the users
learn to voluntarily modify their brain signals, which takes weeks or even
months \cite{wolpaw2002bci}. An alternative, more user friendly approach is to
adapt the \ac{BCI} to the user's naturally occurring brain signals with
\acf{ML} methods (e.g. \cite{vidal1977rtd, blankertz2008osf}), which reduces
the investment of time needed for the first use of a \ac{BCI} from weeks to
minutes. Due to inter-subject variations in the measured neuronal activity
linked to specific mental tasks (the neuronal signature), a \ac{BCI} session
typically starts with a training session in which the subject performs a
known series of mental tasks. After this session, these examples of brain
activity are used to guide the optimization of a classification model that
decodes neuronal activity based on a few seconds of the \ac{EEG} signal. In this
dissertation, we take this latter approach. Note that the two approaches
mentioned above are not mutually exclusive.

% invasive vs non-invasive
The level of invasiveness is another major determining factor in \ac{BCI}
research. In this dissertation, we focus on the use of the electrical signals
that are emitted by large ensembles of neurons, and are \emph{measured on the
outside of the scalp} (i.e. \ac{EEG}). Most \ac{BCI} research in Europe is
based on these external \ac{EEG} measures. Research groups in the United States
generally take a more invasive approach, and place sensors under the scull, or
even deep in the brain. This has the advantage that the measurements are less
contaminated with external noise, and suffer less from spatial smearing
(blurring) by the tissues between the neural sources and the sensors. The
disadvantage is that one needs to undergo surgery, and that the implants
usually work for a limited time. 
%
Less frequently used for \acp{BCI} are non-invasive measures of brain activity
based on magnetic fields (i.e. \acl{MEG}, \acs{MEG}), or on depleting oxygen
concentrations in the blood that indicate brain activity, such as \ac{fMRI} and
\ac{fNIRS} \cite{wolpaw2006bm2}. 

% type of research conducted
Most of the current \ac{BCI} research focusses on better neuronal signatures
(i.e. neuroscience), better decoding of these neural signatures (i.e. signal
processing, machine learning) and on developing specific applications for
patients (e.g. speller grids). The traditionally clinical background of
\ac{BCI} practitioners is reflected in the focus on trial-based discrimination
between a limited and tightly controlled set of tasks. Furthermore, the online
evaluation is often performed in an environment similar to these controlled
off-line experiments, with a classifier that is transplanted from off-line
calibration data and used to classify batches of \ac{EEG} \emph{after} the task
has been performed as indicated by a cue.

\section{BCIs for healthy users}
% HMI/BCI
\begin{sloppypar}
Compared to other groups, the recently started \ac{BCI} research within our
\ac{HMI} group takes a more holistic approach, and attempts to create user
friendly \acp{BCI} based on established neuronal signatures, and evaluate these
in unconstrained environments both on efficacy, user experience and ethical
considerations. \Ac{HMI} aims at \acp{BCI} for healthy users, specifically
applied in gaming contexts. \end{sloppypar}

% gaming
\Ac{BCI} gaming applications are interesting for several reasons.
First, the target population is huge, and gamers are known to be early adopters
of new technology. Second, a less than stellar \ac{BCI} efficacy might not be
problematic in the context of a game, and might even contribute an immersive
challenge (i.e. learn to control a magical in-game construct)
\cite{nijholt2009tsc2}. 
%
If gamers embrace \ac{BCI} technology, this will provides scales to mass-produce
hardware and fund more research, which could eventually lead to feasible
applications outside gaming. However, premature commercial exploitation of
\ac{BCI} technology is feared by the field, as claims of interpreting brain signals are often exaggerated by commercial parties --- which could lead to a disappointed public.

% BCI and gaming challenges
The implied transition from a controlled lab to an unconstrained gaming
environment poses some new challenges. During gaming, signals produced by
facial expressions, speech and eye movement heavily contaminate the, in
comparison weak, \acs{EEG} signals. As such, some of the research at \ac{HMI}
explores the challenges and drawbacks of \ac{BCI} combined with for example
speech recognition \cite{gurkok2010cmi}. Most of our studies allow the user to
behave naturally. The drawback is that this implies careful interpretation of
what measures are based on neuronal signals, and to what extent this is of
importance for the target audience. \todo{MP: Add graphical examples for EEG
and artifacts}

% BCI and gaming opportunities
These challenges of unconstrained environments are balanced by some rather
unique possibilities offered by the \ac{BCI}. For example, the game can use
indicators of imminent movement to anticipate future actions of the user,
thereby blurring the boundary between the user's intentions and explicit
behaviour in interaction. More fundamentally, measures of the user experience
(e.g. workload, attention, or emotional states) can --- if reliable measures
are found --- be used to adapt the game to keep the user in a
state of being fully focused and immersed in the game. This mental state is
known as ``flow'' \cite{csikszentmihalyi1990fpo}. For this reason, some of our
work focusses on automatic recognition of mental states
\cite{reuderink2012vad}. 
 
% measures of attention
Another unique property of \acp{BCI} is that most conventional neural
signatures are related to some form of attention. For example, imaginary
movement produces changes in the \ac{SMR}, and is modulated by attention
\cite{kirsteva-feige2002eap}. Similarly, spelling applications for patients
frequently use the P300 response that is strongly linked to attention
\cite{gray2004pia}. Other examples include the \ac{SSVEP} response to
flickering lights, and direct measures of visual attention
\cite{vangerven2009amp}. This attentional aspect of these neural signatures is
what makes them viable for active \ac{BCI} control.
 
By measuring attention, a \ac{BCI} can offer valuable information that measures
of behaviour never can: it can provide an indication of the \emph{context} that
disambiguates the users behaviour. For example, a user making a phone call
could direct speech commands to the computer if we could detect that the
computer is the object of the user's (covert) attention.

These unique applications depend on a \ac{BCI} that can reliably detect
naturally occurring brain activity. Two key issues in the decoding of brain
signals complicate the development of these applications.
\clearpage % prevents underfull \vbox

\section{Key challenges for BCI adoption}
We identify two key issues that need to be addressed for wide \ac{BCI} adoption
in general: 1) using a \ac{BCI} should be easy, and not require big
investments by the user (e.g. money and time), and 2) the \ac{BCI} should be
dependable, that is to say it should function predictably, with a known
accuracy \cite{millan2010cbc, vangerven2009bci, huggins2011wwb}\footnote{Note
that a known accuracy does not imply that the accuracy has to be high.}. In
addition, the \ac{BCI} should be applied such that it provides something unique
for non-patients, since it cannot (yet) compete on reliability and speed with
existing input devices. 

% Issue 1: Investment of time
Due to the availability of relatively cheap, semi-dry commercial \ac{EEG}
headsets for gaming that show promise for proper \ac{BCI} use
\cite{bobrov2011bci}, the first key issue mainly revolves around investments of
time. The investment of time can be separated into user training and setup
time; both should be kept very short.
%
The maximally acceptable setup time for \ac{ALS} patients is around 30 minutes,
with the maximum of 2--5 sessions in total for user training
\cite{huggins2011wwb}. 
%
While current \acp{BCI} based on \ac{ML} methods can achieve competitive
performance without any user training \cite{vidal1977rtd, blankertz2006bbc},
the time needed to set up the recording equipment and to record a calibration
session typically exceeds this acceptable setup time. Non-patients are probably
even less willing to accept long investments of time. Ideally, we would like to
reduce the setup time to one or two minutes, and remove the calibration and
user training time altogether.

% Issue 2: Unreliability / non-stationary signals
An example of the second key issue is described in the review of
\citet{wolpaw2002bci}, where \ac{ALS} patients in the experiment by
\citet{kuebler1999ttd} report to prefer a slower character-based speller over a
faster word-based speller. They felt more independent since the former was
completely under their control. 
%
Being in control implies that the \ac{BCI} should not behave unexpectedly,
but not necessarily that the \ac{BCI} operates without errors. Lack of errors
is not strictly needed since there is a fundamental trade-off between the
number of errors and the speed of a \ac{BCI} --- the error rate can be reduced
by integrating predictions over a longer period of time. But this trade-off
only holds if the BCI makes mistakes with a constant probability. Therefore, a
\ac{BCI} with a constant error rate should be preferred over a \ac{BCI} with a
variable, but lower error rate, since the former can be relied upon, if slowed
down to acceptable error rates.

Unexpected, erratic \ac{BCI} behaviour can be caused by non-stationary signals.
This is known to be a fundamental problem in the \ac{BCI} field. The inherent
variable nature of spontaneous \ac{EEG} causes changes in the feature
distributions used by the \ac{BCI} to detect and classify brain signals. 
%
One source of this variability in the \ac{EEG} is related to changes in the
user state. For example, differences in levels of alertness, fatigue,
frustration and workload level can alter the characteristics of the \ac{EEG}.
This variability violates basic assumptions made by \ac{ML} methods used to
train the \ac{BCI} classifier, and can result in a loss of performance during
the application of the \ac{BCI} \cite{shenoy2006tac, blankertz2007ics,
krauledat2008tzt}. 
%
Most of the published work on non-stationary signals in \acp{BCI} focusses on
the changing distribution of the \ac{EEG}, and explicitly attempts to reduce
feature variability over time \cite{hill2006tdd, tomioka2006asf,
blankertz2007ics, bunau2009ssa, meinecke2009lis}, or alternatively, to adapt
the classifiers parameters to the changing distribution \cite{shenoy2006tac,
vidaurre2007sol}.
%
An unsolved problem is that it is unclear how variability in the feature
distributions influences the \ac{BCI} performance, since commonly used spatial
filtering methods can already remove task-irrelevant fluctuations to some
degree. In this case, attempting to remove the variability could introduce new
problems that are caused by difficulties in estimating the unrelated
variability in \ac{EEG} features.

Both key issues are interrelated, since they are based on not fully
understood properties of fluctuation in feature distributions over time,
sessions and subjects. In this dissertation, we will investigate the nature of
these variations in the \ac{EEG} distributions, and present two new,
complementary methods that we have devised to overcome the key
issues we have described.

\section{Contributions}
\todo{MP: Add more detail to the contributions}
To ground the problem, we will show in Chapter~\ref{chap:loc} that even changes
in the mental state of the user can induce changes in the \ac{EEG} signals, and
that \acp{BCI} based on commonly used signal features are sensitive to these
changes. 
%
We will proceed in Chapter~\ref{chap:sob} by describing a method aimed at
removing these changes in signal feature distributions. Based on the insight
that a large class of \acp{BCI} are based on \emph{relative changes} in
spectral power, but uses absolute power for classification, we will describe
the new \ac{SOB} features that specifically isolate these changes in neural
firing synchrony, thereby removing long-term and subject-specific deviations.
%
Still, the assumption made by \ac{ML} methods that the training data contains
samples that are \ac{IID} is violated, since samples nearby in time are highly
correlated. This temporal dependence is especially troublesome during the
training of the classifier: due to the overestimation of the amount of
independent information contained in the training set it leads to overfitting.
%
In Chapter~\ref{chap:dsvm} we will present a generalization of the well known
\ac{SVM} classifier, that takes the temporal dependence of features (and hence
the dependence of classification errors) into account. Both on artificial data
and real \ac{BCI} data, overfitting is reduced with this \ac{dSVM}, leading to
an increased information throughput.

